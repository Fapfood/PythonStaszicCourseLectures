\begin{frame}{Definicje funkcji}
    Dzięki zastosowaniu funkcji możemy reużywać fragmenty kodu oraz ukrywać szczegóły implementacyjne za interfejsem. \\
    Deklaracja funkcji (jej interfejs) to jej pierwsza linia (nazwa oraz przyjmowane parametry). Następujący po deklaracji blok kodu to definicja funkcji. \\
    Funkcja może być bezargumentowa lub przyjmować argumenty - ich skończoną lub nieskończoną liczbę. Skończona liczba argumentów może przyjmować wartości domyślne. \\
    Funkcja może zwracać jakąś wartość (słowo kluczowe \emph{return}) lub nie zwracać wartości. Funkcje generujące mogą zwracać wiele wartości (słowo kluczowe \emph{yield}). \\
    \begin{alertblock}{}
        Domyslne argumenty są ewaluowane tylko raz!
    \end{alertblock}
\end{frame}

\begin{frame}{Definicje funkcji}
    \lstinputlisting{introduction/code/functions/functions.py}
\end{frame}

\begin{frame}{Generatory}
    Generatory pozwalają leniwie generować po jednej wartości, bez konieczności przechowywania w pamięci wszystkich przyszłych wartości. \\
    \lstinputlisting{introduction/code/functions/generators.py}
\end{frame}

\begin{frame}{Gorliwy (eager) wujek}
    Funkcje mogą zwracać wartości leniwie lub gorliwie. Gorliwe funkcje od razu zwracają wszystkie wyniki, a leniwe zwracają po jednym za każdym razem gdy się je wywoła. \\
    \begin{exampleblock}{}
       Wujek Janusz jest gorliwy. Gdy proszę go o dowcip każe mi czekać i zapisuje wszystkie znane mu dowcipy na kartce, co trwa kilka godzin. Następnie wręcza mi kartkę i mówi, że gdy najdzie mnie ochota na dowcip, mogę po prostu przeczytać kolejny z kartki. Mam już w domu kilkanaście kartek pełnych tych samych dowcipów wujka Janusza. Taka lista przydała mi się gdy jechałen na kolonie i chciałem zabrać ze sobą wszystkie dowcipy. Czasem jednak chciałbym usłyszeć tylko jeden...
    \end{exampleblock}
\end{frame}

\begin{frame}{Leniwy (lazy) wujek}
    \begin{exampleblock}{}
        Wujek Stefan jest leniwy i zawsze gdy go poproszę opowiada tylko jeden dowcip. Kiedy chciałem spisać wszystkie jego dowcipy, żeby porównać ich listę z listą dowcipów wujka Janusza, musiałem wielokrotnie prosić go o kolejny dowcip i samemu je spisywać. Trwało to kilkanaście godzin, znacznie dłużej niż w przypadku wujka Janusza, który sam je zapisywał. Jednakże, gdy chcę usłyszeć tylko jeden dowcip, wujek Stefan jest znacznie lepszym wyborem niż wujek Janusz.
    \end{exampleblock}
    Generatory są leniwe i odpowiednio użyte mogą znacznie skrócić czas obliczeń. Użyte nieodpowiednio mogą go wydłużyć. \\
\end{frame}

\begin{frame}{Funkcje anonimowe}
    \lstinputlisting{introduction/code/functions/lambda.py}
\end{frame}

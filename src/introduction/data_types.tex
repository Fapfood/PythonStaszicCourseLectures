\begin{frame}{Wbudowane funkcje}
    \Wider{
    \begin{table}
        \centering
        \begin{tabular}{|l|l|}
            \hline
            print() & wypisanie na standardowe wyjście \\
            input() & pobranie ze standardowego wejścia \\
            \hline
            type() & typ zmiennej \\
            \hline
            range() & iterator sekwencji od a do b \\
            \hline
            abs() & wartość bezwzględna \\
            \hline
            min() & najmniejsza wartość sekwencji \\
            max() & największa wartość sekwencji \\
            \hline
            filter() & filtrowanie funkcyjne \\
            map() & mapowanie funkcyjne \\
            \hline
            reversed() & zwrócenie odwróconej sekwencji \\
            sorted() & zwrócenie posortowanej sekwencji \\
            \hline
            $\cdots$ & $\cdots$ \\
            \hline
        \end{tabular}
    \end{table}
    }
    \url{https://docs.python.org/3.7/library/functions.html} - wszystkie dostępne
\end{frame}

\begin{frame}{Typy danych}
    W Pythonie mamy wiele wbudowanych typów danych:
    \begin{itemize}
        \item float() - reprezentuje liczby rzeczywiste \footnote{Nie jest to do końca prawda, o czym przekonasz się na kolejnych slajdach.} \\
        \item int() - reprezentuje liczby całkowite \\
        \item str() - reprezentuje napisy - łańcuchy znaków \\
        \item bool() - reprezentuje wartości logiczne \\
        \item complex() - reprezentuje liczby urojone \\
        \item list() - reprezentuje zmienną grupę danych \\
        \item tuple() - reprezentuje uporządkowaną strukturę danych \\
        \item dict() - reprezentuje grupę danych, które da się indeksować \\
        \item set() - reprezentuje zbiór danych w rozumieniu teorii zbiorów \\
    \end{itemize}
    \url{https://docs.python.org/3.7/library/stdtypes.html} - wszystkie dostępne
\end{frame}

\begin{frame}{Klasyfikacja typów danych}
    \begin{columns}
        \begin{column}{0.5\textwidth}
            Typy skalarne:
            \begin{itemize}
                \item float() \\
                \item int() \\
                \item bool() \\
                \item str() - jako znak np. dla funkcji ord() \\
            \end{itemize}
            Typy drzewiaste:
            \begin{itemize}
                \item dict() \\
                \item set() \\
            \end{itemize}
            Typy liczbowe:
            \begin{itemize}
                \item float() \\
                \item int() \\
                \item complex() \\
            \end{itemize}            
        \end{column}
        \begin{column}{0.5\textwidth}
            Typy strukturalne:
            \begin{itemize}
                \item complex() \\
                \item str() - jako łańcuch znaków \\
                \item list() \\
                \item tuple() \\
                \item dict() \\
                \item set() \\
            \end{itemize}
            Typy sekwencyjne:
            \begin{itemize}
                \item str() - jako łańcuch znaków \\
                \item list() \\
                \item tuple() \\
            \end{itemize}
        \end{column}
    \end{columns}
\end{frame}

\begin{frame}{Przypisanie}
    \lstinputlisting{introduction/code/data_types/assignment.py}
\end{frame}
